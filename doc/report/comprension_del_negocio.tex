\subsection{Determinar los objetivos del negocio}
    \subsubsection{Escenario actual}
    Al momento de escribir este informe, la empresa en cuestión tiene interés
    en desarrollar un canal de youtube, y requiere medir de alguna forma las
    probabilidades de éxito de llevar a cabo videos populares.
    \subsubsection{Objetivos del negocio}
    El objetivo es en este caso poder  cuáles son las condiciones que un canal
    de youtube y sus videos debe cumplir para que los mismos sean exitosos en
    la plataforma.
    \subsubsection{Criterios de éxito del negocio}
    El proyecto se considerará exitoso si se llegan a detectar las variables
    clave que influyen en que los videos del canal de youtube sean exitosos o
    no, y en qué nivel influye cada una de ellas. Se considerará que los videos
    del canal de youtube si los mismos obtienen un porcentaje de X likes y más
    de 2 millones de vistas por video.
\subsection{Evaluación de la situación}
    \subsubsection{Inventario de recursos}
    \subsubsection{Requisitos, supuestos y restricciones}
        \paragraph{Requisitos}
        \paragraph{Supuestos}
        \paragraph{Restricciones}
    \subsubsection{Riesgos y contingencias}
    \subsubsection{Terminología}
        \paragraph{Glosario de términos del negocio}
        \paragraph{Glosario de términos de la minería de datos}
    \subsubsection{Costos y beneficios}
\subsection{Determinar objetivos de Minería de Datos}
    \subsubsection{Objetivo de minería de datos}
    \subsubsection{Criterios de éxito de minería de datos}
\subsection{Realizar el Plan del Proyecto}
    \subsubsection{Plan de proyecto}
    \subsubsection{Validación inicial de las herramientas}