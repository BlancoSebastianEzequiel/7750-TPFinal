\subsection{Determinar los objetivos del negocio}
    \subsubsection{Escenario actual}
    Al momento de escribir este informe, la empresa en cuestión tiene interés
    en desarrollar un canal de youtube, y requiere medir de alguna forma las
    probabilidades de éxito de llevar a cabo videos populares.
    \subsubsection{Objetivos del negocio}
    El objetivo es en este caso poder  cuáles son las condiciones que un canal
    de youtube y sus videos debe cumplir para que los mismos sean exitosos en
    la plataforma.
    \subsubsection{Criterios de éxito del negocio}
    El proyecto se considerará exitoso si se llegan a detectar las variables
    clave que influyen en que los videos del canal de youtube sean exitosos o
    no, y en qué nivel influye cada una de ellas. Se considerará que un video es
    exitoso si los mismos obtienen una cantidad de vistas mayor a 2 millones
    por video.

\subsection{Evaluación de la situación}
    \subsubsection{Inventario de recursos}
    Para desarrollar el presente proyecto se cuenta con una amplia gama de recursos que
    asegura un desarrollo de calidad y confianza del mismo.
    Se cuenta con un set de datos extraído de la pagina de kaggle sobre videos
    de youtube que fueron subidos a la plataforma. Además se cuenta con
    herramientas de software de análisis y visualización de datos líderes
    en el mercado, como Pandas. Por último, se cuenta con personal
    altamente calificado para la correcta interpretación de los mismos.

    \subsubsection{Requisitos, supuestos y restricciones}
        \paragraph{Requisitos}
        Contar con datos suficientes y sobre todo representativos de la
        plataforma youtube
        \paragraph{Supuestos}
        Los datos en estudio son lo suficientemente correctos como para poder
        sacar conclusiones confiables a partir de ellos.
        \paragraph{Restricciones}
        Se cuenta solamente con datos de videos subidos a youtube
        Estados Unidos. No se asegura que los resultados sean válidos para
        otros países.

    \subsubsection{Riesgos y contingencias}

        Si bien se puede llevar a cabo un análisis lo más riguroso posible, siempre existe la
        posibilidad de que un video pueda no ser exitoso porque hay que tener en
        cuenta que el exito de cada video puede depender de muchos factores de
        incertidumbre, y si bien se cumplen patrones, no hay nada que asegure al
        100 \% el éxito de los mismos.
        En caso de detectar que un video no está teniendo el éxito esperado,
        habrá que recurrir a las diversas métricas que pueda ofrecer Youtube a
        desarrolladores y analizar la situación

    \subsubsection{Terminología}

        \paragraph{Glosario de términos del negocio}

            \underline{\textbf{Youtube:}} es una plataforma que te permite subir videos a
            partir de la creación de un canal.\\\\
            \underline{\textbf{Canal de youtube:}} Es el equivalente a crearse una
            cuenta en cierta aplicacion, en la cual se almacenan los videos subidos\\\\
            \underline{\textbf{Categoría:}} define el nombre de un grupo de vides con
            cualidades comunes.\\\\
            \underline{\textbf{likes/dislikes:}} Es un atributo de un video que muestra la
            cantidad de usuarios que le dieron like al video en cuestion\\\\
            \underline{\textbf{Vistas:}} Es la cantidad de veces que el video fue
            visto\\\\
            \underline{\textbf{Video exitoso:}} Aquel cuya cantidad de vistas
            es mayor a 200000 \\\\

        \paragraph{Glosario de términos de la minería de datos}

            \underline{\textbf{Atributo:}}  dato sobre alguna característica de las
            observaciones.\\\\
            \underline{\textbf{Atributo relevante:}} atributo que juega un papel principal
            en la clasificación, por lo que la clase dependerá en alguna medida de
            qué valor tenga.\\\\
            \underline{\textbf{Registro:}} fila que representa una observación, está
            compuesto de atributos.\\\\
            \underline{\textbf{Dataset:}} conjunto de datos a ser utilizados para la
            ejecución de los algoritmos de Data Mining, está compuesto de registros.\\\\
            \underline{\textbf{Filtrado de atributos:}} específica al dataset formado
            considerando sólo los atributos relevantes.\\\\
            \underline{\textbf{Regla:}} es una implicación, que representa una acción
            mediante una condición. Sigue la estructura “Si\..., entonces\...”.\\\\
            \underline{\textbf{Soporte:}} es la relación entre la cantidad total de
            registros del dataset que cumplen la regla y la cantidad de observaciones
            procesadas.\\\\
            \underline{\textbf{Confianza:}} es la relación entre la cantidad total de
            observaciones de la clase mayoritaria que cumplen la regla y la cantidad
            de observaciones que fueron afectadas por esa misma regla.\\\\
            \underline{\textbf{Captura:}} es la relación entre la cantidad de
            observaciones de la clase mayoritaria que cumplen la regla y la cantidad de
            observaciones procesadas pertenecientes a esa misma clase.\\\\
            \underline{\textbf{Coeficiente de correlación de Pearson:}} es la estadística
            de prueba que mide la relación estadística, o asociación, entre dos
            variables continuas. Es conocido como el mejor método para medir la
            asociación entre variables de interés porque se basa en el método de
            covarianza. Da información sobre la magnitud de la asociación, o
            correlación, así como la dirección de la relación\\\\

    \subsubsection{Costos y beneficios}
        El beneficio del proyecto es detectar las características que hacen a
        un video de youtube sea exitoso. De esta manera se pueden tener en
        cuenta ciertos parametros para poder desarrollar un video que exitoso
        basandose en la historia.
        Al ser un trabajo final educativo, no hay costos.
\subsection{Determinar objetivos de Minería de Datos}
    \subsubsection{Objetivo de minería de datos}
    El objetivo de minería de datos es el análisis de los datos obtenidos a
    partir de la información disponible, buscando obtener así información
    relevante que permita predecir las condiciones bajo las cuales una
    aplicación es exitosa.
    \subsubsection{Criterios de éxito de minería de datos}
    Selección de al menos 4 reglas con captura y soporte mayor o igual al 3 \%
    y con una confianza mayor al 60 \%
\subsection{Realizar el Plan del Proyecto}
    \subsubsection{Plan de proyecto}
        \begin{itemize}
            \item Recolección de datos: 5 horas.
            \item Preparación de datos: 5 horas.
            \item Ejecución del algoritmo de Inducción: 4 horas.
            \item Análisis de resultados de algoritmo de Inducción: 4 horas.
            \item Combinación de resultados: 3 horas.
            \item Elaboración de reporte: 7 horas.
        \end{itemize}
    \subsubsection{Validación inicial de las herramientas}
        Se utilizarán las siguientes herramientas:
        \begin{itemize}
            \item Python
            \item Pandas
            \item Weka
            \item Jupiter Notebook
            \item Numpy
        \end{itemize}
