\subsection{Seleccionar los datos}
    \subsubsection{Inclusión/Exclusión de datos}
        Los datos que se terminaron utilizando fueron todos los de tipo
        alfanumerico, numérico y discreto.
\subsection{Limpiar los datos}
    \subsubsection{Reporte de limpieza de datos}
        Se limpiaron los siguientes campos:
        \begin{itemize}
            \item video\_id
            \item trending\_date
            \item title
            \item channel\_title
            \item tags
            \item likes
            \item dislikes
            \item thumbnail\_link
            \item description
        \end{itemize}
        Algunos de estos campos se usaron para crear nuevos campos.
\subsection{Estructurar los datos}
    \subsubsection{Derivación de atributos}
        \begin{itemize}
            \item \textbf{days\_since\_publication:} Cantidad de dias desde que se publico el video
            \item \textbf{title\_size:} Longitud en cantidad de caracteres del atributo title
            \item \textbf{publish\_time:} Se modifico el timestamp por un formato tal: YYYY-MM-DD
            \item \textbf{tags\_quantity:} cantidad de tags
            \item \textbf{likes\_ratio:} Ratio de likes y dislikes calculado como: likes/(likes+dislikes)
            \item \textbf{has\_description:} Es verdadero si el video tiene descripcion.
        \end{itemize}
    \subsubsection{Generación de registros}
        Se genero el registro \code{is\_sucessfull} que fue calculado en funcion de si
        el video tiene mas de 2 millones de viws y un ratio mayor o igual a 0.8
\subsection{Integrar los datos}
    \subsubsection{Unificación de los datos}
        Se unieron las datos del dataset con el json de categorias para conocer
        la relacion entre el valor nuemrico y el nombre de la categoria.
\subsection{Formato de los datos}
    \subsubsection{Reporte de formato de los Datos}
        publish\_time: Se modifico el timestamp por un formato tal: YYYY-MM-DD