\subsection{Reporte de calidad de los datos}

    A continuacion mostraremos un cuadro que muestra que atributos obtuvimos y
    de que tipo son:\\\\
    \resizebox{\columnwidth}{!}{\begin{tabular}{||c | c | c||}
        \hline
        \textbf{Nombre de variable} & \textbf{tipo de dato} & \textbf{Descripcion} \\ [0.5ex]
        \hline\hline
        category\_id & numeric & son cuatro numeros que representan un grupo de categorias \\
        \hline
        publish\_time & date & momento exacto de la publicacion del video \\
        \hline
        comments\_disabled & booleano & Dice si tiene o no los comentarios habilitados \\
        \hline
        ratings\_disabled & booleano & Dice si tiene o no los likes/dislikes habilitados \\
        \hline\
        video\_error\_or\_removed & booleano & Dice si el video sufrio un error o fue borrado en el momento de la fecha de trending\_date \\
        \hline
        title\_size & numeric & cantidad de caracteres que tiene el titulo del video \\
        \hline
        tags\_quantity & numeric & cantidad de etiquetas que el video tiene \\
        \hline
        likes\_ratio & numeric & porcentaje de likes respecto de dilikes \\
        \hline
        has\_description & booleano & Dice si el video tiene descripcion o no \\
        \hline
        comments\_per\_view & numeric & Cantidad de cantidad de comentarios dividido cantidad de views \\
        \hline
        progess & numeric & Promedio de vistas segun pasan los dias desde la publicacion del video \\
        \hline
        is\_successful & boolean & Dice si el video es exitoso o no \\
        \hline
    \end{tabular}}

\newpage
\subsection{Seleccionar los datos}
    \subsubsection{Inclusión/Exclusión de datos}
        Los datos que se terminaron utilizando fueron todos los de tipo
        alfanumerico, numérico y discreto.

\subsection{Limpiar los datos}
    \subsubsection{Reporte de limpieza de datos}
        Se limpiaron los siguientes campos:
        \begin{itemize}
            \item video\_id
            \item trending\_date
            \item title
            \item channel\_title
            \item tags
            \item likes
            \item dislikes
            \item thumbnail\_link
            \item description
            \item comment\_count
        \end{itemize}
        Algunos de estos campos se usaron para crear nuevos campos
        .
\subsection{Estructurar los datos}
    \subsubsection{Derivación de atributos}
        \begin{itemize}
            \item \textbf{title\_size:} Longitud en cantidad de caracteres del atributo title
            \item \textbf{publish\_time:} Se modifico el timestamp por un formato tal: YYYY-MM-DD
            \item \textbf{tags\_quantity:} cantidad de tags
            \item \textbf{likes\_ratio:} Ratio de likes y dislikes calculado como: likes/(likes+dislikes)
            \item \textbf{has\_description:} Es verdadero si el video tiene descripcion.
            \item \textbf{comments\_per\_view:} Cantidad de comentarios dividido la cantidad de vistas del video
            \item \textbf{progess:} Promedio de vistas segun pasan los dias desde la publicacion del video
        \end{itemize}
    \subsubsection{Generación de registros}
        Se genero el registro \code{is\_sucessfull} que fue calculado en funcion de si
        el video tiene mas de 2 millones de views

\subsection{Integrar los datos}
    \subsubsection{Unificación de los datos}

        Se unieron las datos del dataset con el json de categorias para conocer
        la relacion entre el valor nuemrico y el nombre de la categoria pero como
        el data ser ya venia con el valor numerico no fue necesario integrar los
        datos ya que nosotros necesitamos el valor numerico. En cambio, esto fue
        necesario para el analisis ya nos sirve para saber de que categorias
        hablamos. Pero luego hicimos una discretizacion de grupos de categorias,
        ya que nos parecia mas importante sea por cantidad de ocurrencias de
        cada una o por importancia.
        Por lo tanto a continuacion mostramos el diccionario discretiado que nos
        dice a que grupo de categorias pertenece cada numero\\

        \paragraph{Diccionario de categorias}

            \begin{center}
                \begin{tabular}{||c | c||}
                    \hline
                    \textbf{Nombre} & \textbf{Category\_id} \\ [0.5ex]
                    \hline\hline
                    Film and Animation & 0 \\
                    \hline
                    Gaming or Music & 1 \\
                    \hline
                    Videoblogging or People and Blogs orComedy or Entertainment & 2 \\
                    \hline
                    News and Politics or Education & 3 \\
                    \hline
                    others & 4 \\
                    \hline
                \end{tabular}
            \end{center}

\subsection{Formato de los datos}
    \subsubsection{Reporte de formato de los Datos}
        publish\_time: Se modifico el timestamp por un formato tal: YYYY-MM-DD
