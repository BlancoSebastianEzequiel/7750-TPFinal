\subsection{Reporte de calidad de los datos}
    A continuacion mostraremos un cuadro que muestra que atributos obtuvimos y
    de que tipo son:\\
    \resizebox{\columnwidth}{!}{\begin{tabular}{||c | c | c||}
        \hline
        \textbf{Mombre de variable} & \textbf{tipo de dato} & \textbf{Descripcion} \\ [0.5ex]
        \hline\hline
        category\_id & numeric & it is the number that represent a category nameo \\
        \hline
        publish\_time & date & momento exacto de la publicacion del video \\
        \hline
        comments\_disabled & booleano & Dice si tiene o no los comentarios habilitadoso \\
        \hline
        ratings\_disabled & booleano & Dice si tiene o no los likes/dislikes habilitadoso \\
        \hline\
        video\_error\_or\_removed & booleano & Dice si el video sufrio un error o fue borrado en el momento de la fecha de trending\_dateo \\
        \hline
        days\_since\_publication & numeric & Cantidad de dias desde que se publico el video \\
        \hline
        title\_size & numeric & cantidad de caracteres que tiene el titulo del video \\
        \hline
        tags\_quantity & numeric & cantidad de etiquetas que el video tiene \\
        \hline
        likes\_ratio & numeric & porcentaje de likes respecto de dilikes \\
        \hline
        has\_description & booleano & Dice si el video tiene descripcion o no \\
        \hline
        comments\_per\_view & numeric & Cantidad de cantidad de comentarios dividido cantidad de views \\
        \hline
    \end{tabular}}
\newpage
\subsection{Seleccionar los datos}
    \subsubsection{Inclusión/Exclusión de datos}
        Los datos que se terminaron utilizando fueron todos los de tipo
        alfanumerico, numérico y discreto.
\subsection{Limpiar los datos}
    \subsubsection{Reporte de limpieza de datos}
        Se limpiaron los siguientes campos:
        \begin{itemize}
            \item video\_id
            \item trending\_date
            \item title
            \item channel\_title
            \item tags
            \item likes
            \item dislikes
            \item thumbnail\_link
            \item description
            \item comment\_count
        \end{itemize}
        Algunos de estos campos se usaron para crear nuevos campos.
\subsection{Estructurar los datos}
    \subsubsection{Derivación de atributos}
        \begin{itemize}
            \item \textbf{days\_since\_publication:} Cantidad de dias desde que se publico el video
            \item \textbf{title\_size:} Longitud en cantidad de caracteres del atributo title
            \item \textbf{publish\_time:} Se modifico el timestamp por un formato tal: YYYY-MM-DD
            \item \textbf{tags\_quantity:} cantidad de tags
            \item \textbf{likes\_ratio:} Ratio de likes y dislikes calculado como: likes/(likes+dislikes)
            \item \textbf{has\_description:} Es verdadero si el video tiene descripcion.
            \item \textbf{comments\_per\_view:} Cantidad de comentarios dividido la cantidad de vistas del video
        \end{itemize}
    \subsubsection{Generación de registros}
        Se genero el registro \code{is\_sucessfull} que fue calculado en funcion de si
        el video tiene mas de 2 millones de viws y un ratio mayor o igual a 0.8
\subsection{Integrar los datos}
    \subsubsection{Unificación de los datos}
        Se unieron las datos del dataset con el json de categorias para conocer
        la relacion entre el valor nuemrico y el nombre de la categoria.
\subsection{Formato de los datos}
    \subsubsection{Reporte de formato de los Datos}
        publish\_time: Se modifico el timestamp por un formato tal: YYYY-MM-DD
