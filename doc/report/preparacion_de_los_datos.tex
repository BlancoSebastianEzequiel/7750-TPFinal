\subsection{Reporte de calidad de los datos}

    A continuacion mostraremos un cuadro que muestra que atributos obtuvimos y
    de que tipo son:\\\\
    \resizebox{\columnwidth}{!}{\begin{tabular}{||c | c | c||}
        \hline
        \textbf{Mombre de variable} & \textbf{tipo de dato} & \textbf{Descripcion} \\ [0.5ex]
        \hline\hline
        category\_id & numeric & it is the number that represent a category nameo \\
        \hline
        publish\_time & date & momento exacto de la publicacion del video \\
        \hline
        comments\_disabled & booleano & Dice si tiene o no los comentarios habilitadoso \\
        \hline
        ratings\_disabled & booleano & Dice si tiene o no los likes/dislikes habilitadoso \\
        \hline\
        video\_error\_or\_removed & booleano & Dice si el video sufrio un error o fue borrado en el momento de la fecha de trending\_dateo \\
        \hline
        days\_since\_publication & numeric & Cantidad de dias desde que se publico el video \\
        \hline
        title\_size & numeric & cantidad de caracteres que tiene el titulo del video \\
        \hline
        tags\_quantity & numeric & cantidad de etiquetas que el video tiene \\
        \hline
        likes\_ratio & numeric & porcentaje de likes respecto de dilikes \\
        \hline
        has\_description & booleano & Dice si el video tiene descripcion o no \\
        \hline
        comments\_per\_view & numeric & Cantidad de cantidad de comentarios dividido cantidad de views \\
        \hline
    \end{tabular}}

\newpage
\subsection{Seleccionar los datos}
    \subsubsection{Inclusión/Exclusión de datos}
        Los datos que se terminaron utilizando fueron todos los de tipo
        alfanumerico, numérico y discreto.

\subsection{Limpiar los datos}
    \subsubsection{Reporte de limpieza de datos}
        Se limpiaron los siguientes campos:
        \begin{itemize}
            \item video\_id
            \item trending\_date
            \item title
            \item channel\_title
            \item tags
            \item likes
            \item dislikes
            \item thumbnail\_link
            \item description
            \item comment\_count
        \end{itemize}
        Algunos de estos campos se usaron para crear nuevos campos
        .
\subsection{Estructurar los datos}
    \subsubsection{Derivación de atributos}
        \begin{itemize}
            \item \textbf{days\_since\_publication:} Cantidad de dias desde que se publico el video
            \item \textbf{title\_size:} Longitud en cantidad de caracteres del atributo title
            \item \textbf{publish\_time:} Se modifico el timestamp por un formato tal: YYYY-MM-DD
            \item \textbf{tags\_quantity:} cantidad de tags
            \item \textbf{likes\_ratio:} Ratio de likes y dislikes calculado como: likes/(likes+dislikes)
            \item \textbf{has\_description:} Es verdadero si el video tiene descripcion.
            \item \textbf{comments\_per\_view:} Cantidad de comentarios dividido la cantidad de vistas del video
        \end{itemize}
    \subsubsection{Generación de registros}
        Se genero el registro \code{is\_sucessfull} que fue calculado en funcion de si
        el video tiene mas de 2 millones de views y un ratio mayor o igual a 0.8

\subsection{Integrar los datos}
    \subsubsection{Unificación de los datos}

        Se unieron las datos del dataset con el json de categorias para conocer
        la relacion entre el valor nuemrico y el nombre de la categoria pero como
        el data ser ya venia con el valor numerico no fue necesario integrar los
        datos ya que nosotros necesitamos el valor numerico. En cambio, esto fue
        necesario para el analisis ya nos sirve para saber de que categorias
        hablamos. Por lo tanto a continuacion mostramos el diccionario que nos
        dice a que categoria pertenece cada numero\\
        \newpage
        \paragraph{Diccionario de categorias}

            \begin{center}
                \begin{tabular}{||c | c||}
                    \hline
                    \textbf{Nombre} & \textbf{Category\_id} \\ [0.5ex]
                    \hline\hline
                    Film and Animation & 1 \\
                    \hline
                    Autos and Vehicles & 2 \\
                    \hline
                    Music & 10 \\
                    \hline
                    Pets and Animals & 15 \\
                    \hline
                    Sports & 17 \\
                    \hline
                    Short Movies & 18 \\
                    \hline
                    Travel and Events & 19 \\
                    \hline
                    Gaming & 20 \\
                    \hline
                    Videoblogging & 21 \\
                    \hline
                    People and Blogs & 22 \\
                    \hline
                    Comedy & 23 \\
                    \hline
                    Entertainment & 24 \\
                    \hline
                    News and Politics & 25 \\
                    \hline
                    Howto and Style & 26 \\
                    \hline
                    Education & 27 \\
                    \hline
                    Science and Technology & 28 \\
                    \hline
                    Nonprofits and Activism & 29 \\
                    \hline
                    Movies & 30 \\
                    \hline
                    Anime-Animation & 31 \\
                    \hline
                    Action-Adventure & 32 \\
                    \hline
                    Classics & 33 \\
                    \hline
                    Comedy & 34 \\
                    \hline
                    Documentary & 35 \\
                    \hline
                    Drama & 36 \\
                    \hline
                    Family & 37 \\
                    \hline
                    Foreign & 38 \\
                    \hline
                    Horror & 39 \\
                    \hline
                    Sci-Fi-Fantasy & 40 \\
                    \hline
                    Thriller & 41 \\
                    \hline
                    Shorts & 42 \\
                    \hline
                    Shows & 43 \\
                    \hline
                    Trailers & 44 \\
                    \hline
                \end{tabular}
            \end{center}

\subsection{Formato de los datos}
    \subsubsection{Reporte de formato de los Datos}
        publish\_time: Se modifico el timestamp por un formato tal: YYYY-MM-DD
