\subsection{Evaluar Resultado}
    \subsubsection{Valoración de los resultados de minería de datos}

        Luego de haber aplicado las técnicas introducidas previamente, se han
        llegado a inferir algunas reglas que dejan al descubierto algunas
        caracterśticas que aumentan la probabilidad de éxito de un video.
        Si bien queda lugar a más análisis debido a que con las reglas no se
        llega a cubrir un panorama completo de las cuestiones a tener en cuenta
        en el desarrollo de un video de youtube, se puede confirmar que se ha
        obtenido información que de otra forma sería difícil obtener. Ademas
        que los valores que ponemos a continuacion son solo aproximaciones y
        que simplemente deben ser usados no en forma estricta, sino como guía
        para mejorar las chances de que el video sea exitoso.\\
        En general, revisando las reglas encontramos algunos patrones en común
        que categorizarían a un video como exitosa:
        \begin{itemize}
            \item El video tiene que tener un progreso de entre 268248 y 1912737 vistas por dia
            \item Tiene que tener categorias como cine, animacion, juegos, musica, comedia, entretenimiento
            \item La cantidad de tags tienen que ser alrededor de
            \item El largo del titulo tiene que ser de menor a 21
            \item Se tiene un entre un 80 y 96 porciento de likes respecto de dislikes
        \end{itemize}
        Por todo esto, se puede considerar que el proyecto de investigación fue
        exitoso.
    \subsubsection{Modelo aprobado}

        El árbol nos dió información útil. Cabe notar que visualizando el mismo
        se obtuvo un mejor panorama de los criterios buscados.

\subsection{Proceso de revisión}
    \subsubsection{Revisión del proceso}

        A partir del análisis exploratorio de datos, seguido de la generación
        de reglas, se ha llegado a cumplir con el objetivo planteado, por lo
        que en principio no quedan actividades pendientes.

\subsection{Determinar Próximos pasos}
    \subsubsection{Listado de posibles acciones}

        Como próximos pasos se podría dejar algún análisis más profundo en
        bases de datos de otros paises y no restringirse solo a la de Estados
        Unidos, ya que se supone que las reglas de éxito puede variar de país
        en país. Esto podría dar reglas muchísimo más claras.