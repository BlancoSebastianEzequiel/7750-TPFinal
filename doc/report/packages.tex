\documentclass[a4paper, 12pt]{article}
\usepackage[T1]{fontenc}
\usepackage[scale=1,angle=0,opacity=1,color=black!60]{background}
\usepackage{tikzpagenodes}
\usepackage{lastpage}
\usepackage{tabularx}
\usepackage{graphicx}
\usepackage{flafter}
\usepackage{lmodern}
\usepackage{float}
\usepackage{adjustbox}
\usepackage{amsmath}
\usepackage{nccmath}
\usepackage{url}
\usepackage{listings}
\usepackage{alltt}
\usepackage[section]{placeins}
\usepackage{tocloft}
\usepackage{hyperref}
\usepackage[textwidth=420pt,textheight=630pt]{geometry}
\usepackage[spanish, activeacute]{babel} %Definir idioma español
\usepackage[utf8]{inputenc} %Codificacion utf-8
\usepackage{titlesec}
\setcounter{secnumdepth}{4}

\hypersetup{
    colorlinks,
    citecolor=black,
    filecolor=black,
    linkcolor=black,
    urlcolor=blue
}
\renewcommand{\cftsecleader}{\cftdotfill{\cftdotsep}}  % lineas punteadas en la tabla de contenidos
\setlength{\oddsidemargin}{15.5pt}
\backgroundsetup{contents={}} %Saca el 'draft'
\definecolor{mygray}{rgb}{0.95,0.95,0.95}
\lstset{
    basicstyle=\footnotesize,
    backgroundcolor=\color{mygray},
    breaklines=true,
    breakatwhitespace=true,
    postbreak=\mbox{\textcolor{red}{$\hookrightarrow$}\space},
    captionpos=b,
    keepspaces=true,
    numbers=left,
    numbersep=5pt,
    showspaces=false,
    showstringspaces=false,
    showtabs=false,
    tabsize=4,
    language=C,
    frame=none,
    title=\lstname,
}
\def\labelitemi{$\bullet$}

% Esto lo saque de aca: https://tex.stackexchange.com/questions/60209/how-to-add-an-extra-level-of-sections-with-headings-below-subsubsection
\titleformat{\paragraph}
{\normalfont\normalsize\bfseries}{\theparagraph}{1em}{}
\titlespacing*{\paragraph}
{0pt}{3.25ex plus 1ex minus .2ex}{1.5ex plus .2ex}

\titleformat{\subparagraph}
{\normalfont\normalsize\bfseries}{\theparagraph}{1em}{}
\titlespacing*{\subparagraph}
{0pt}{3.25ex plus 1ex minus .2ex}{1.5ex plus .2ex}

% Esto lo saque de aca: https://tex.stackexchange.com/questions/36030/how-to-make-a-single-word-look-as-some-code
% Better inline directory listings
\usepackage{xcolor}
\definecolor{light-gray}{gray}{0.95}
\newcommand{\code}[1]{\colorbox{light-gray}{\texttt{#1}}}
